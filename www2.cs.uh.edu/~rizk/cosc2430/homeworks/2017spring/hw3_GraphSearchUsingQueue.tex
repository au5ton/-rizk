\documentclass[times,11pt,verbatim,js-singlespace]{article}  % 10pt default?
\usepackage[left=1in,right=1in,top=1in,bottom=0.8in,footskip=.25in]{geometry}
\usepackage{algorithm2e}
\usepackage{graphicx}
\usepackage{color}
\title{COSC2430: Programming and Data Structures\\
Homework 4: Graph Algorithm using Queue
}

\date{}

\begin{document}
\special{papersize=8.5in,11in} 

\pagestyle{plain}
\let\thepage\relax  % no page number

\maketitle

\section{Introduction}
Reachability from a source vertex $s$ is the problem to find the set of vertices $S$ such that $v \in S$ if exists a path from $s$ to $v$. You will create a C++ program to find reachable vertices using \textcolor{red}{Breath First Search (BFS) algorithm}. 
\section{Input and Output}
$G = (V, E)$ is a directed graph, where $n$ vertices and $m$ edges. $G$ can be represented as an adjacency matrix $E, n\times n$, where $n <= 100$. Please see Figure 1 as an example. You will read a sparse matrix $E$ from an input file; There will be ONE matrix entry per line in the input file and each line will have a triplet of integer numbers $i,j,v$ where $1 \le i,j \le n$ indicate the entry and $v\neq 0$ indicates a directed edge pointing from vertex $i$ to $j$. Given a source vertex $1 \le s \le n$, your program should display all reachable vertices \textcolor{red}{(excluding the source vertex $s$) in ascending order}. 
\begin{figure}[htbp]
\centering\includegraphics[width=.6\textwidth]{graph.png}
\label{F:GraphAndE}
\caption{The adjacency matrix $E$  (sparse representation) for a sample graph $G$}
\end{figure}

Input example for Figure 1 (the last line indicates the matrix dimension $n = 11$).
\begin{verbatim}
2 1 1
2 3 1
3 5 1
4 5 1
5 6 1
5 7 1
5 8 1
6 4 1
7 6 1
9 5 1
10 9 1
11 2 1
11 11 0
\end{verbatim}

Output example (source=2)

\begin{verbatim}
1 3 4 5 6 7 8
\end{verbatim}

\section{Program and input specification}

The main program should be called "reachability".
Call syntax:
\begin{verbatim}
reachability "E=<input_file>;sourse=<num>".
\end{verbatim}

\section{Requirements}
\begin{itemize}
\item \textcolor{red}{std::queue or std::deque} is not allowed to used. You should implement your own queue data structure.
\item You program should get the result within 10 seconds.
\end{itemize}
\end{document}

