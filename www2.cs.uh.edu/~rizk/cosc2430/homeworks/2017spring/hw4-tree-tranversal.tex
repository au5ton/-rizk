
%\documentclass[times,11pt,verbatim,js-singlespace, xcolor, graphicx,wrapfig]{article}  % 10pt default?
%\usepackage[dvipsnames]{}

\documentclass[times,11pt,verbatim,js-singlespace]{article}  % 10pt default?
\usepackage[left=1in,right=1in,top=1in,bottom=0.8in,footskip=.25in]{geometry}
\usepackage{wrapfig}
\usepackage{graphicx}
\usepackage{color}
\title{COSC2430: Programing and Data Structures\\
Binary Tree Traversals
}

%\author{Instructor: Carlos Ordonez}
\date{}

\begin{document}

\pagestyle{plain}
\let\thepage\relax  % no page number

\maketitle

\section{Introduction}
You will create a C++ program that can display the post-order traversal of a binary tree given the pre-order and in-order traversals of the tree.
\section{Input and Output}
The input are two regular text files, one contains the pre-order traversal of a tree and another contains the in-order. In the file, positive infinite integers (less than 1000 integers) are separated with one or more space characters. \textcolor{red}{You cannot assume the tree is balanced. You cannot assume the binary tree is binary search tree. But you can assume no duplicated numbers are in the files.} The program should display the post-order to the screen. 

%\begin{figure}[htbp]
%  \centering
%  \includegraphics[width=.45\textwidth]{tree.png}
%  \caption{Tree}
%\end{figure}

\begin{wrapfigure}{r}{0.4\textwidth}
\hspace{-0.2in}
    \includegraphics[width=0.4\textwidth]{tree}
\end{wrapfigure}
Input example: preorder1.txt
\begin{verbatim}
5   2 1 4 3 7 6 8
\end{verbatim}
Input example: inorder1.txt
\begin{verbatim}
1 2 3 4 5 6  7 8
\end{verbatim}
Post-order output example:
\begin{verbatim}
1 3 4 2 6 8 7 5
\end{verbatim}

\section{Program input specification}
The main program should be called ”traversal”. The output should be written to the console (e.g.
printf or cout), but the TAs will redirect it to create some output file.

Call syntax at the OS prompt (notice double quotes):
\begin{verbatim}
traversal "preorder=<filename>;inorder=<filename>"
\end{verbatim}

Example of program call:
\begin{verbatim}
./traversal "preorder=preorder1.txt;inorder=inorder1.txt"
\end{verbatim}
\section{Requirements}
\begin{itemize}
\item You can use std::string or other data structures to represent the infinite integer.
\item Your program should get the result within 10 seconds.
\end{itemize}
\end{document}

