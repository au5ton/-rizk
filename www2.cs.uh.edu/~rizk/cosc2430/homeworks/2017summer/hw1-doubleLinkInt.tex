
\documentclass[times,11pt,verbatim,js-singlespace]{article}  % 10pt default?
\usepackage[left=1in,right=1in,top=1in,bottom=0.8in,footskip=.25in]{geometry}
\usepackage{algorithm2e}
\usepackage{graphicx}
\usepackage{color}

\title{COSC2430: Programming and Data Structures\\
Doubly Linked Lists and Infinite Precision Arithmetic
}

%\author{Instructor: Carlos Ordonez}
\date{}

\begin{document}

\pagestyle{plain}
\let\thepage\relax  % no page number

\maketitle

\section{Introduction}

You will create a C++ program that
can evaluate arithmetic operators with integer numbers having any number of digits.
These numbers are an alternative to fixed size integers or floating
point numbers that always have a maximum number of accurate digits 
(dependent on size of CPU register).

\section{Input and Output}

The input is a regular text file, where each line
is terminated with an end-of-line character(s).
Each line will contain an arithmetic operation
between two numbers.
The program should display the input expression adn the results, separated with =.

Input example:
\begin{verbatim}
0*0
0+1
123456*2593
2*2000000000000000
2*3
1+10
10000000000000000+1
12345667890123456789+8765432109876543210
999999999999999999999+1
\end{verbatim}

Output example:
\begin{verbatim}
0*0=0
0+1=1
123456*2593=320121408
2*2000000000000000=4000000000000000
2*3=6
1+10=11
10000000000000000+1=10000000000000001
1234567890123456789+8765432109876543210=9999999999999999999
999999999999999999999+1=1000000000000000000000
\end{verbatim}



Advanced operators and functions example:
\begin{verbatim}
sqrt(16)=4
sqrt(81)=9
20/6=3,remainder=2
1500/5=300
sqrt(13489)=116.1421543
1000000000000/2=500000000000
\end{verbatim}



\section{Program input and output specification}

The main program should be called "infinitearithmetic".
The output should be written to the console (e.g. printf or cout),
but the TAs will redirect it to create some output file.

Call syntax at the OS prompt (notice double quotes):\\
\begin{verbatim}
infinitearithmetic input=<file name> digitsPerNode=<number>
\end{verbatim}

Assumptions:
\begin{itemize}
\item The file is a small plain text file (say $< 10000$ bytes); no need to handle binary files.
\item Only integer numbers as input (no decimals!). Output number without leading zeroes.
\item Operators: $ + * $
\item 
      do not break an arithmetic expression into multiple lines as it will mess testing.
% there may be a space included for clarity between a number and the operator.
%      You can also use 1 space to separate operators and numbers,
\end{itemize}


Example of program call:
\begin{verbatim}
infinitearithmetic input=xyz.txt digitsPerNode=2
\end{verbatim}



\section{Requirements}

\begin{itemize}
\item Correctness is the most important requirement: TEST
 your program with many input files.
 Your program should not crash or produce exceptions.

\item Doubly linked lists are required. A program using
arrays to store long numbers will receive a failing grade (below 50).
However, arrays for parameters or other auxiliary variables are acceptable.

\item Breaking a number into a list of nodes. Each node will store the number of digits specified in
the parameters. Notice it is acceptable to "align" digits after reading the entire number so 
that that the rightmost node (end) has all the digits.

Example of numbers stored as a list of nodes of 2 digits:
\begin{verbatim}
963 stored as {9,63} or {96,3}
123456 stored as {12,34,56}
0 stored as {0}
\end{verbatim}


\item Doubly linked list features:

Numbers must be read and inserted manipulating the list forward
starting with the most significant digit (leftmost digit).
Each node must be multiplied by some power of 10, depending on its
position within the list. Assuming 2-digit nodes the powers
would be 1,100,10000,..

\item Arithmetic operators:

Addition: Numbers must be added  starting on the least significant digit,
keeping a carryover from node to node.

Multiplication: numbers must be multiplied with the traditional method you learned
in elementary school starting from the rightmost digit. 
However, your multiplication
algorithm must handle the number of digits per node specified,
one of which may be the simplest case you already know: 1 digit per node.
You must use linked lists to store partial results.

\item input and output numbers.

The input numbers must be stored on lists.
The result of the arithmetic operation must be stored on a third list as well.
Printing must be done traversing the list forward.
After the addition is complete and the result is printed
to the output file the program must deallocate
(delete) the lists.
\textcolor{red}{Remove leading zeroes from the result (e.g. 1, instead of 00001; 0 instead of 00000)}.

\item Memory allocation and deallocation.

Your program must use "new" to allocated each node and must free memory (using the "delete" operator).
at the end.


\item Limits:
Each node will store a fixed number of digits, specified with the "digits per node" parameter.
You can assume 2-8 digits per node. \textcolor{red}{If you use 1 digit per node or don't use the parameter specified from the command line, your grade will have 50\%'s penalty.}

You can assume input text lines can have up to 256 (2**8) characters,
but that should not be a limit in the list.
You should not assume a maximum number of lines for the input file 
(e.g. a file may hav many blank lines).

\item Advanced operators and functions (optional):
Subtraction, allowing negative numbers as result, no credit.
\textcolor{red}{Division with remainder, harder and 10\% credit.
Square root function $sqrt(number)$ including 20 decimals, hardest, 20\% extra credit.}

\item \textcolor{red}{Your program should get the result within 2 seconds}.
\end{itemize}


% \bibliography{db}

\end{document}

